\usepackage{iftex}
\ifPDFTeX
    \usepackage[utf8]{inputenc}
    \usepackage[T1]{fontenc}
    \usepackage{lmodern}
    \usepackage[ngerman]{babel}
\fi
\ifLuaTeX
    \usepackage{fontspec}
    \setmainfont{CMU Serif}
    \usepackage{polyglossia}
    \setdefaultlanguage[spelling=new,babelshorthands=true]{german}
\fi
\ifXeTeX
    \usepackage{fontspec}
    \setmainfont{CMU Serif}
    \usepackage{polyglossia}
    \setdefaultlanguage[spelling=new,babelshorthands=true]{german}
\fi
\usepackage{microtype}
\usepackage{beamerthemedefault}

% footer stuff
\definecolor{mycolor}{HTML}{e2001a}
\setbeamercolor{title in head/foot}{bg=mycolor, fg=white}
\setbeamercolor{structure}{fg=mycolor}

% slightly modified from beamerouterthememiniframes.sty
\setbeamertemplate{footline}{%
    \begin{beamercolorbox}[ht=2.5ex,dp=1.125ex,%
      leftskip=.3cm,rightskip=.3cm plus1fil]{title in head/foot}%
      {\usebeamerfont{title in head/foot}\insertsectionhead
      \hfill
      \insertframenumber/\inserttotalframenumber
      }%
    \end{beamercolorbox}%
    \begin{beamercolorbox}[colsep=1.5pt]{lower separation line foot}
    \end{beamercolorbox}
}

\logo{
    \includegraphics[height=10mm]{img/logos/dhbw}
    \vspace{200pt}
}
\title{Präsentationsvorlage Kolloquium}
\subtitle{Untertitel}
\author{Kümmerlin, Jonas}
\date{Ein Datum}
\subject{Projektarbeit}

\begin{document}

% actual content
\frame{\titlepage}
\section{Inhalt}
\begin{frame}
    \frametitle{Inhalt}
    \begin{enumerate}
        \item Eins
        \item Zwei
        \item Oder drei?
    \end{enumerate}
\end{frame}
\section{Ein Abschnitt}
\begin{frame}
    \frametitle{Stichpunkte}
    \begin{itemize}
        \item<+-> Das ist eine Präsentation
        \item<+-> Verwendet wird die \texttt{beamer}-Klasse \cite{tantau2010latex}
        \item<+-> PowerPoint ist oftmals auch eine gute Option
    \end{itemize}
\end{frame}
\section{Quellen}
\begin{frame}[allowframebreaks]
    \frametitle{Quellenverzeichnis}
    \bibliographystyle{apalike}
    \bibliography{references/bibliothek}
\end{frame}
\end{document}
